\documentclass[10pt]{beamer}

\usepackage[spanish, mexico]{babel}
\usepackage[utf8]{inputenc}

\usetheme[progressbar=frametitle]{metropolis}
\usepackage{appendixnumberbeamer}

\usepackage{booktabs}
\usepackage[scale=2]{ccicons}

\usepackage{pgfplots}
\usepgfplotslibrary{dateplot}

\usepackage{xspace}
\newcommand{\themename}{\textbf{\textsc{metropolis}}\xspace}

%%
\usepackage{color}
\definecolor{lstgrey}{rgb}{0.95,0.95,0.95}
\definecolor{mygreen}{RGB}{28,172,0} % color values Red, Green, Blue
\definecolor{mylilas}{RGB}{170,55,241}

\usepackage{listings}
\lstset{language=Python,
       backgroundcolor=\color{lstgrey},
       frame=single,
       basicstyle=\footnotesize\ttfamily,
       captionpos=b,
       tabsize=2,
  }

\lstset{language=Python,%
  %basicstyle=\color{red},
  breaklines=true,%
  morekeywords={python2tikz},
  keywordstyle=\color{blue},%
  morekeywords=[2]{1}, keywordstyle=[2]{\color{black}},
  identifierstyle=\color{black},%
  stringstyle=\color{mylilas},
  commentstyle=\color{mygreen},%
  showstringspaces=false,%without this there will be a symbol in the places where there is a space
  numbers=left,%
  numberstyle={\tiny \color{black}},% size of the numbers
  numbersep=9pt, % this defines how far the numbers are from the text
  emph=[1]{for,end,break},emphstyle=[1]\color{red}, %some words to emphasise
  %emph=[2]{word1,word2}, emphstyle=[2]{style},    
}
%

\lstset{language=C,
       backgroundcolor=\color{lstgrey},
       frame=single,
       basicstyle=\footnotesize\ttfamily,
       captionpos=b,
       tabsize=2,
  }

\lstset{language=C,%
  %basicstyle=\color{red},
  breaklines=true,%
  morekeywords={c2tikz},
  keywordstyle=\color{blue},%
  morekeywords=[2]{1}, keywordstyle=[2]{\color{black}},
  identifierstyle=\color{black},%
  stringstyle=\color{mylilas},
  commentstyle=\color{mygreen},%
  showstringspaces=false,%without this there will be a symbol in the places where there is a space
  numbers=left,%
  numberstyle={\tiny \color{black}},% size of the numbers
  numbersep=9pt, % this defines how far the numbers are from the text
  emph=[1]{for,end,break},emphstyle=[1]\color{red}, %some words to emphasise
  %emph=[2]{word1,word2}, emphstyle=[2]{style},    
}
%


\title{Introducción a la programación con Python 3}
\subtitle{Conceptos fundamentales}
\date{\today}
% \date{}
\author{Ing. Jose Eduardo Laruta Espejo}
\institute{Universidad La Salle}
% \titlegraphic{\hfill\includegraphics[height=1.5cm]{../img/lasalle}}

\begin{document}

\maketitle

\begin{frame}[allowframebreaks]{Contenido}
  \setbeamertemplate{section in toc}[sections numbered]
  \tableofcontents[]
\end{frame}

%%%

\section{Estructuras de datos}
\subsection{Listas}

\begin{frame}[fragile]{Listas}
    \begin{itemize}
        \item Las \alert{Listas} almacenan una secuencia de objetos mutables.
        \item Los objetos almacenados no tienen que ser del mismo tipo.
    \end{itemize}

\begin{lstlisting}
>>> frutas = ['manzana', 'durazno', 'pera', 'banana']
>>> frutas[0]
'manzana'
>>> otras_frutas = ['naranja', 'papaya', 'kiwi']
>>> frutas + otras_frutas
['manzana', 'durazno', 'pera', 'banana', 'naranja', 'papaya', 'kiwi']
\end{lstlisting}

\end{frame}

\begin{frame}[fragile]{Listas}
    Se puede acceder a un elemento de una lista usando el operador \texttt{[]}

\begin{lstlisting}
>>> frutas[0]
'manzana'
>>> frutas[2]
'pera'
>>> frutas[-1]
'banana'
\end{lstlisting}

\end{frame}

\begin{frame}[fragile]{Listas}
Se puede acceder a varios elementos contiguos con el operador slice
usando la siguiente forma general: \texttt{frutas[inicio, fin]} va a 
retornar los elementos en los indices \texttt{inicio, inicio+1, ..., fin-1}.

\begin{lstlisting}
>>> frutas[0:2]
['manzana', 'durazno']
>>> frutas[:3]
['manzana', 'durazno', 'pera']
>>> frutas[2:]
['pera', 'banana']
\end{lstlisting}

\end{frame}

\begin{frame}[fragile]{Listas}
Las listas pueden contener cualquier objeto de Python. Podemos 
tener listas de listas.
\begin{lstlisting}
>>> matriz = [[1, 2, 3], [4, 5, 6]]
>>> matriz[1][2]
6
>>> lista = ['hola', 2, ['a', 2, 'adios'], 45]
>>> lista[-1]
45
\end{lstlisting}
Ejercicio: explorar métodos de una lista con el comando \texttt{dir(lista)} y 
consultar su documentación con \texttt{help(lista.metodo)}
\end{frame}


\begin{frame}[fragile]{Tuplas}
Las tuplas son estructuras similares a las listas con la diferencia de 
que son inmutables, es decir, sus elementos no se pueden cambiar luego de 
crearse la tupla.
\begin{lstlisting}
>>> par = (3, 4)
>>> par[0]
3
>>> x, y = par
>>> y
4
>>> par[1] = 6
\end{lstlisting}
Ejercicio: explorar métodos de una lista con el comando \texttt{dir(lista)} y 
consultar su documentación con \texttt{help(lista.metodo)}
\end{frame}

\begin{frame}[fragile]{Conjuntos(Sets)}
Los conjuntos (Sets) son estructuras no ordenadas que no pueden tener items duplicados.
Se puede crear un Set a partir de una lista.
\begin{lstlisting}
>>> formas = ['circulo', 'cuadrado', 'rombo', 'triangulo']
>>> set_formas = set(formas)
>>> set_formas2 = {'circulo', 'cuadrado', 'rombo', 'triangulo'}
\end{lstlisting}
Se pueden aplicar operaciones comunes de conjuntos (Unión, intersección) a los 
sets de Python.
\end{frame}

\begin{frame}[fragile]{Diccionarios}
Una de las estructuras más útiles de Python es el Diccionario, el cual almacena 
un mapeo de un tipo de datos (key) a otro tipo (value). El key debe ser un tipo inmutable, 
mientras que el value puede ser cualquier tipo de python.
\begin{lstlisting}
>>> estudiante = {'nombre':'jose', 'edad':22, 'notas':[78, 90, 88]}
>>> estudiante['notas']
[78, 90, 88]
>>> estudiante.keys()
['nombre', 'edad', 'notas']
>>> estudiante.values()
['jose', 22, [78. 90, 88]]
>>> estudiante.items()
[('nombre','jose'), ('edad',22), ('notas',[78, 90, 88])]
>>> len(estudiante)
3
\end{lstlisting}
Al igual que con las listas anidadas, se pueden crear diccionaros de diccionarios.
\end{frame}



\begin{frame}[fragile]{Scripts de python}
El intérprete de python puede ejecutar comandos de forma interactiva, pero para 
tareas de programación más serias y completas usaremos archivos de código fuente 
denominados \texttt{scripts}.
Un script es un conjunto de sentencias de python que se ejecutan de forma ordenada.

\begin{lstlisting}
# esto es un comentario
frutas = ['manzana', 'durazno', 'pera', 'banana']
for fruta in frutas:
    print(fruta + ' a la venta')
frutasdic = ['manzana':15, 'durazno':20, 'pera':10, 'banana':5]
for fruta, precio in frutasdic.items():
    if precio < 12:
        print('el precio de la {} es {}'.format fruta, precio)
    else:
        print('{} es una fruta muy cara'.format(fruta))
\end{lstlisting}
\end{frame}

\begin{frame}[fragile]{Scripts de python}
Para ejecutar un script de python se debe invocar al intérprete 
ingresando el nombre del archivo como argumento.
\begin{lstlisting}
$ python frutas.py
\end{lstlisting}
\end{frame}

\begin{frame}[fragile]{List Comprehension}
Se puede construir una lista en python usando una forma de notación especial que se  define en 
una expresión dentro de la lista.
\begin{lstlisting}
nums = [1,2,3,4,5,6]
masuno = [x + 1 for x in nums]
impares = [x for x in nums if x % 2 == 1]
impares_mas_uno = [x + 1 for x in nums if x % 2 == 1]
\end{lstlisting}
\end{frame}

\begin{frame}[fragile]{List Comprehension (Ejercicio)}
Dada la siguiente lista, escriba un list comprehension que genere una versión en 
minúsculas de cada cadena que tenga longitud mayor que 5.
\begin{lstlisting}
nums = ['Hola',
        'escuDEro',
        'Buenos Dias',
        'ADIOS',
        'RinoceronTE',
        'algoritmos',
        'Inteligencia Artificial']
\end{lstlisting}
\end{frame}

\begin{frame}[fragile]{Funciones}
Al igual que en Java o en C++, en Python podemos declarar nuestras propias 
funciones:
\begin{lstlisting}
frutas = ['manzana':15, 'durazno':20, 'pera':10, 'banana':5]

def comprarFruta(fruta, kilos):
    if fruta not in frutas:
        print('Lo siento, no contamos con {}'.format fruta)
    else:
        costo = frutas[fruta] * kilos
        print('Son {} Bolivianos, por favor'.format(costo))
    
comprarFruta('pera', 2)
comprarFruta('mango', 10)
\end{lstlisting}
\end{frame}



\begin{frame}[fragile]{Clases y Objetos}
Un objeto encapsula datos y provee funciones para interactuar con los datos. En 
Python se pueden definir objetos personalizados usando Clases.
\begin{lstlisting}
class Perro:
    def __init__(self, nombre, raza, edad):
        self.nombre = nombre 
        self.raza = raza 
        self.edad = edad 
    def ladrar(self):
        print('guau')
    def saludar(self):
        print('guau, soy {} y soy de raza {}'.format(self.nombre, self.raza))
\end{lstlisting}
Encapsular datos previene su uso inadecuado. 

La abstracción de objeto permite escribir código más sencillo de entender.
\end{frame}

\begin{frame}[fragile]{Clases y Objetos}
Para aprovechar las ventajas de los objetos se debe instanciar los mismos 
en código.
\begin{lstlisting}
cachuchin = Perro('cachuchin', 'chapi', 9)
cachuchin.ladrar()
cachuchin.saludar()

lassie = Perro('lassie', 'collie', 7)
lassie.saludar()
\end{lstlisting}
Encapsular datos previene su uso inadecuado. 

La abstracción de objeto permite escribir código más sencillo de entender.
\end{frame}

\section{Práctica de Laboratorio 1}

\begin{frame}[fragile]{Laboratorio 1}
El Laboratorio 1 tiene como objetivo introducir a la programación con Python, 
se cuenta con un proyecto autoevaluado sobre el cual el estudiante tendrá que realizar
modificaciones segun los ejercicios.
El archivo con lo necesario se encuentra en \href{https://inst.eecs.berkeley.edu/~cs188/fa18/assets/files/tutorial.zip}{este link}
\end{frame}

\begin{frame}[fragile,allowframebreaks]{Instrucciones}
Una vez descargado el archivo, descomprimir el contenido en alguna carpeta destinada a 
proyectos de programación. dentro de la carpeta encontrará varios archivos de python.

Para este Laboratorio, se tienen un conjunto de archivos que deberán editar:
\begin{itemize}
    \item \texttt{addition.py}: ejercicio 1
    \item \texttt{buyLotsOfFruit.py}: ejercicio 2
    \item \texttt{shop.py}: ejercicio 3
    \item \texttt{shopSmart.py}: ejercicio 3
\end{itemize}
Una vez editados los archivos correspondientes a los ejercicios, deberá ejecutar el autoevaluador 
para comprobar la solución.

El archivo para evaluar los ejercicios se llama \texttt{autograder.py}

Para ejecutar la evaluación de los ejercicios:
\begin{lstlisting}
$ python autograder.py
\end{lstlisting}
Los archivos restantes se pueden ignorar, no los toque.

\end{frame}

\begin{frame}[fragile,allowframebreaks]{Ejercicio 1 (ejemplo)}
Abrir el archivo \texttt{addition.py} y ver la definición de \texttt{add}:
\begin{lstlisting}
def add(a, b):
    "Return the sum of a and b"
    "*** YOUR CODE HERE ***"
    return 0
\end{lstlisting}
Modificar de la siguiente manera:
\begin{lstlisting}
def add(a, b):
    "Retorna la suma de a y b"
    print('a={}, b={}, retornando a+b={}'.format(a,b,a+b))
    return a+b
\end{lstlisting}
Correr a evaluacion:
\begin{lstlisting}
$ python autograder.py -q q1
\end{lstlisting}
\end{frame}



\begin{frame}[fragile]{Ejercicio 2}
Añadir una función llamada \texttt{buyLotsOfFruit(orderList)} al archivo \texttt{buyLotsOfFruit.py}
que tome como parámetro una lista de tuplas \texttt{(fruta, kilo)} y retorne el costo 
de la lista. Si aparece una fruta en la lista que no este presente en \texttt{fruitPrices} se debe 
imprimir un mensaje de error y retornar \texttt{None}. 

No cambie la variable \texttt{fruitPrices}
\end{frame}

\begin{frame}[fragile]{Ejercicio 3}
Complete la función \texttt{shopSmart(orders,shops)} en el archivo \texttt{shopSmart.py}
que recibe un \texttt{orderList} (similar al del anterior ejercicio) y una lista 
de objetos \texttt{FruitShop} y retorne el objeto \texttt{FruitShop} que posea el menor costo 
total. No cambie los nombres de archivos o de variables.

Compruebe el funcionamiento usando el archivo de evaluación \texttt{autograder.py}

\end{frame}

\appendix

% \begin{frame}[fragile]{Backup slides}
%   Sometimes, it is useful to add slides at the end of your presentation to
%   refer to during audience questions.

%   The best way to do this is to include the \verb|appendixnumberbeamer|
%   package in your preamble and call \verb|\appendix| before your backup slides.

%   \themename will automatically turn off slide numbering and progress bars for
%   slides in the appendix.
% \end{frame}


\end{document}
