\documentclass[10pt]{beamer}

\usepackage[spanish, mexico]{babel}
\usepackage[utf8]{inputenc}

\usetheme[progressbar=frametitle]{metropolis}
\usepackage{appendixnumberbeamer}
\usepackage{amsfonts}
\usepackage{amssymb}
\usepackage{amsmath}

\usepackage{booktabs}
\usepackage[scale=2]{ccicons}

\usepackage{tikz}
\def\checkmark{\tikz\fill[scale=0.4](0,.35) -- (.25,0) -- (1,.7) -- (.25,.15) -- cycle;}
\usepackage{pgfplots}
\usepgfplotslibrary{dateplot}

\usepackage{xspace}
\newcommand{\themename}{\textbf{\textsc{metropolis}}\xspace}

%%
\usepackage{color}
\definecolor{lstgrey}{rgb}{0.95,0.95,0.95}
\definecolor{mygreen}{RGB}{28,172,0} % color values Red, Green, Blue
\definecolor{mylilas}{RGB}{170,55,241}

\usepackage{listings}
\lstset{language=Python,
       backgroundcolor=\color{lstgrey},
       frame=single,
       basicstyle=\footnotesize\ttfamily,
       captionpos=b,
       tabsize=2,
  }

\lstset{language=Python,%
  %basicstyle=\color{red},
  breaklines=true,%
  morekeywords={python2tikz},
  keywordstyle=\color{blue},%
  morekeywords=[2]{1}, keywordstyle=[2]{\color{black}},
  identifierstyle=\color{black},%
  stringstyle=\color{mylilas},
  commentstyle=\color{mygreen},%
  showstringspaces=false,%without this there will be a symbol in the places where there is a space
  numbers=left,%
  numberstyle={\tiny \color{black}},% size of the numbers
  numbersep=9pt, % this defines how far the numbers are from the text
  emph=[1]{for,end,break},emphstyle=[1]\color{red}, %some words to emphasise
  %emph=[2]{word1,word2}, emphstyle=[2]{style},    
}
%

\lstset{language=C,
       backgroundcolor=\color{lstgrey},
       frame=single,
       basicstyle=\footnotesize\ttfamily,
       captionpos=b,
       tabsize=2,
  }

\lstset{language=C,%
  %basicstyle=\color{red},
  breaklines=true,%
  morekeywords={c2tikz},
  keywordstyle=\color{blue},%
  morekeywords=[2]{1}, keywordstyle=[2]{\color{black}},
  identifierstyle=\color{black},%
  stringstyle=\color{mylilas},
  commentstyle=\color{mygreen},%
  showstringspaces=false,%without this there will be a symbol in the places where there is a space
  numbers=left,%
  numberstyle={\tiny \color{black}},% size of the numbers
  numbersep=9pt, % this defines how far the numbers are from the text
  emph=[1]{for,end,break},emphstyle=[1]\color{red}, %some words to emphasise
  %emph=[2]{word1,word2}, emphstyle=[2]{style},    
}
%


\title{ISI306 - Aprendizaje Automático}
\subtitle{Repaso de álgebra lineal}
\date{\today}
% \date{}
\author{Ing. Jose Eduardo Laruta Espejo}
\institute{Universidad La Salle - Bolivia}
% \titlegraphic{\hfill\includegraphics[height=1.5cm]{logo.pdf}}

\begin{document}

\maketitle

\begin{frame}[allowframebreaks]{Contenido}
  \setbeamertemplate{section in toc}[sections numbered]
  \tableofcontents[]
\end{frame}

%%%



\section{Conceptos básicos y Notación}
\begin{frame}
    \frametitle{Conceptos básicos y Notación}
    El campo del álgebra lineal nos presenta formas muy eficientes de 
    representar y realizar operaciones con grupos de números ordenados o conjuntos
    de ecuaciones lineales, por ejemplo:
    
    \begin{align*}
        4x_1 - 5x_2 &= -13 \\
        -2x_1 + 3x_2 &= 9
    \end{align*}

    El mismo conjunto de ecuaciones se puede representar en forma matricial $ Ax = b$ siendo:
    \begin{align*}
        A = \begin{bmatrix}
            4 & -5 \\
            -2 & 3
        \end{bmatrix}
        , 
        b = \begin{bmatrix}
            -13 \\ 
            9
        \end{bmatrix}
    \end{align*}
\end{frame}

\begin{frame}
    \frametitle{Notación}
    Se usará la siguiente notación:
    \begin{itemize}
        \item $A \in \mathbb{R}^{m\times n}$ denota una matriz con $m$ filas y $n$ columnas, donde los elementos de $A$ son reales.
        \item $x \in \mathbb{R}^n$ denota un vector con $n$ elementos. Por convención un vector $n$-dimensional también se representa como una matriz con $n$ filas y 1 columna.
        \item El $i$ésimo elemento de un vector $x$ se denota como $x_i$.
            \begin{equation}
                x = \begin{bmatrix}
                    x_1 \\
                    x_2 \\
                    \vdots \\
                    x_n
                \end{bmatrix}
            \end{equation}
    \end{itemize}
\end{frame}

\begin{frame}
    \frametitle{Notación}
    \begin{itemize}
        \item $a_{ij}$ (o $A_{ij}, A_{i,j}$) es un elemento de $A$ en la fila $i$ y la columna $j$:
            \begin{equation}
                A = \begin{bmatrix}
                    a_{11} & a_{12} & \cdots & a_{1n} \\
                    a_{21} & a_{12} & \cdots & a_{2n} \\
                    \vdots & \vdots & \ddots & \vdots \\
                    a_{m1} & a_{m2} & \cdots & a_{mn} 
                \end{bmatrix}
            \end{equation}
        \item $a_{j}$ o $A_{:,j}$ es la $j$-ésima columna de la matriz $A$:
            \begin{equation}
                A = \begin{bmatrix}
                    | & | &   & | \\
                    a_1 & a_2 & \cdots & a_n \\
                    | & | &   & | 
                \end{bmatrix}
            \end{equation}
        \item $a_{i}^T$ o $A_{i,:}$ es la $i$-ésima fila de la matriz $A$:
            \begin{equation}
                A = \begin{bmatrix}
                    \text{\textemdash} & a_1^T & \text{\textemdash} \\
                    \text{\textemdash} & a_2^T & \text{\textemdash} \\
                          & \vdots &  \\
                    \text{\textemdash} & a_m^T & \text{\textemdash} \\
                \end{bmatrix}
            \end{equation}
    \end{itemize}
\end{frame}


\section{Multiplicación de matrices}

\begin{frame}
    \frametitle{Multiplicación de matrices}
    Dadas las matrices $A \in \mathbb{R}^{m\times n}$ y $B \in \mathbb{R}^{n\times p}$ el 
    producto matricial es:
    \begin{equation}
        C = AB \in \mathbb{R}^{m \times p}
    \end{equation}
    donde:
    \begin{equation}
        C_{ij}= \sum_{k = 1}^n A_{ik}B_{kj}
    \end{equation}
    nótese que, para que el producto matricial exista, el número de columnas de $A$ debe 
    ser igual al número de filas de $B$. En tal caso se denominan matrices conformables.

\end{frame}

\begin{frame}
    \frametitle{Producto de vectores}
    Dados dos vectores $x,y \in \mathbb{R}^n$, el producto $x^T y$ se suele llamar el \textbf{producto interno}
    o \textbf{producto escalar}, el resultado es un escalar:
    
    \begin{equation}
        x^Ty \in \mathbb{R}^n = [x_1 \ x_2 \ \cdots \ x_n]\begin{bmatrix}
            y_1\\
            y_2\\
            \vdots \\
            y_n
        \end{bmatrix} = \sum_{i=1}^n x_i y_i
    \end{equation}
    El producto escalar es solamente un caso especial de un producto matricial.
    
\end{frame}

\begin{frame}
    \frametitle{Producto de vectores}
    Por otro lado, dados 2 vectores $x \in \mathbb{R}^m, y \in \mathbb{R}^n$ (no del mismo tamaño necesariamente)
    el resultado de $xy^T$ se conoce como el \textbf{producto externo} o \textbf{producto vectorial} y 
    es una matriz dada por $(xy^T)_{ij} = x_iy_j$

    \begin{equation}
        xy^T \in \mathbb{R}^n = \begin{bmatrix}
            x_1\\
            x_2\\
            \vdots \\
            x_n
        \end{bmatrix} [y_1 \ y_2 \ \cdots \ y_n]= \begin{bmatrix}
            x_1y_1 & x_1y_2 & \cdots & x_1y_n \\
            x_2y_1 & x_2y_2 & \cdots & x_2y_n \\
            \vdots & \vdots & \ddots & \vdots \\
            x_m1y_1 & x_my_2 & \cdots & x_my_n
        \end{bmatrix}
    \end{equation}

    Esta operación puede util para distintos casos, por ejemplo: repetir un vector en columnas de una matriz.
\end{frame}

\begin{frame}
    \frametitle{Producto matriz-vector}
    Dada la matrices $A \in \mathbb{R}^{m\times n}$ y un vector $x \in \mathbb{R}^n$, el producto da 
    como resultado otro vector $y = Ax \in \mathbb{R}^m$

    \begin{equation}
        y = Ax = A = \begin{bmatrix}
            \text{\textemdash} & a_1^T & \text{\textemdash} \\
            \text{\textemdash} & a_2^T & \text{\textemdash} \\
                  & \vdots &  \\
            \text{\textemdash} & a_m^T & \text{\textemdash} \\
            \end{bmatrix} x = \begin{bmatrix}
             a_1^Tx \\
             a_2^Tx \\
             \vdots \\
             a_m^Tx
        \end{bmatrix}
    \end{equation}

    En este caso, el $i$-ésimo elemento de $y$ es el producto escalar de la $i$-ésima fila de $A$ 
    y el vector $x$. También se puede hacer una pre multiplicación por un vector.
\end{frame}

\begin{frame}
    \frametitle{Producto de matrices}
    Entendiendo los productos entre vectores, el producto matricial se puede expresar en función 
    de distintos productos escalares, dadas las matrices
    $A \in \mathbb{R}^{m\times n}$ y $B \in \mathbb{R}^{n\times p}$:

    \begin{equation}
        C = AB = \begin{bmatrix}
            \text{\textemdash} & a_1^T & \text{\textemdash} \\
            \text{\textemdash} & a_2^T & \text{\textemdash} \\
                  & \vdots &  \\
            \text{\textemdash} & a_m^T & \text{\textemdash}
            \end{bmatrix} \begin{bmatrix}
                | & | &   & | \\
                b_1 & b_2 & \cdots & b_p \\
                | & | &   & | 
            \end{bmatrix} = 
            \begin{bmatrix}
                a_1^Tb_1 & a_1^Tb_2 & \cdots & a_1^Tb_p \\
                a_2^Tb_1 & a_2^Tb_2 & \cdots & a_2^Tb_p \\
                \vdots & \vdots & \ddots & \vdots \\
                a_m^Tby_1 & a_m^Tb_2 & \cdots & a_m^Tb_p
            \end{bmatrix}
    \end{equation}
\end{frame}

\begin{frame}
    \frametitle{Propiedades}
    Es importante construir un entendimiento intuitivo de las operaciones entre matrices y vectores 
    por su importancia en el desarrollo de los algoritmos de aprendizaje automático que se explorarán 
    en el futuro y en su implementación en python. Adicionalmente, la multiplicación de matrices 
    posee las siguientes propiedades:
    \begin{itemize}
        \item Asociatividad: $(AB)C = A(BC)$
        \item Distributividad: $A(B + C) = AB + AC$
        \item No es conmutativa: $AB \neq BA$
    \end{itemize}
    

\end{frame}

\section{Operaciones y propiedades}
\begin{frame}
    \frametitle{Matriz identidad}
    La \textbf{matriz identidad}, denotada por $I \in \mathbb{R}^{n\times n}$, es una matriz cuadrada
    con unos en la diagonal y ceros en otras posiciones:
    \begin{equation}
        I_{ij} = \begin{cases}
            1 & i = j\\
            0 & i \neq j
        \end{cases}
    \end{equation}
    y tiene la propiedad que para toda matriz $A \in \mathbb{R}^{m\times n}$,
    \begin{equation}
        AI = A = IA
    \end{equation}
    Generalmene, las dimensiones de la matriz identidad $I$ se infieren del contexto en el que se usa.
\end{frame}

\begin{frame}
    \frametitle{Matriz Diagonal}
    En una \textbf{matriz diagonal} todos los elementos que no pertenecen a la diagonal son iguales a 0. 
    Se suele denotar como $D = diag(d_1, d_2, \dots, d_n)$, donde:
    \begin{equation}
        D_{ij} = \begin{cases}
            d_i & i = j \\
            0 & i \neq j
        \end{cases}
    \end{equation}
    Claramente, $I = diag(1, 1, \dots, 1)$
\end{frame}

\begin{frame}
    \frametitle{Matriz Traspuesta}
    La traspuesta de una matriz resulta de ``intercambiar'' las filas con las columnas. Dada una matriz 
    $A \in \mathbb{R}^{m\times n}$, su traspuesta $A^T \in \mathbb{R}^{n\times m}$ es una matriz con dimensiones
    $n \times m$ donde sus elementos estan definidos por:
    \begin{equation}
        (A^T)_{ij} = A_{ji}
    \end{equation}
    Tiene las siguientes propiedades:
    \begin{itemize}
        \item $(A^T)^T = A$
        \item $(AB)^T = B^TA^T$
        \item $(A + B)^T = A^T + B^T$
    \end{itemize}
    Una matriz cuadrada es \textbf{simétrica} si $A = A^T$. Y es \textbf{antisimétrica} si $A = -A^T$.
\end{frame}

\begin{frame}
    \frametitle{Matriz inversa}
    La inversa de una matriz $A \in \mathbb{R}^{m\times n}$, denotada por $A^{-1}$ y cumple>:
    \begin{equation}
        A^{-1}A = I = AA^{-1}
    \end{equation}
    No todas las matrices tienen inversa, por ejemplo, matrices no cuadradas, e incluso en el caso 
    de algunas matrices cuadradas, la inversa puede no existir. Si $A$ posee inversa se dice que 
    es una matriz \textbf{invertible} o \textbf{no singular}, en otro caso es \textbf{no invertible}
    o \textbf{singular}
\end{frame}

\begin{frame}
    \frametitle{Matriz inversa}
    La inversa de una matriz $A \in \mathbb{R}^{m\times n}$, denotada por $A^{-1}$ y cumple>:
    \begin{equation}
        A^{-1}A = I = AA^{-1}
    \end{equation}
    No todas las matrices tienen inversa, por ejemplo, matrices no cuadradas, e incluso en el caso 
    de algunas matrices cuadradas, la inversa puede no existir. Si $A$ posee inversa se dice que 
    es una matriz \textbf{invertible} o \textbf{no singular}, en otro caso es \textbf{no invertible}
    o \textbf{singular}
\end{frame}

% \begin{frame}
%     \frametitle{Determinante}
%     El determinante de una matriz cuadrada $A \in \mathbb{R}^{n\times n}$, es una función 
%     $det : \mathbb{R}^{n \times n} \leftarrow \mathbb{R}$ denotada por $|A|$ o $det(A)$
%     \begin{equation}
%         A^{-1}A = I = AA^{-1}
%     \end{equation}
%     No todas las matrices tienen inversa, por ejemplo, matrices no cuadradas, e incluso en el caso 
%     de algunas matrices cuadradas, la inversa puede no existir. Si $A$ posee inversa se dice que 
%     es una matriz \textbf{invertible} o \textbf{no singular}, en otro caso es \textbf{no invertible}
%     o \textbf{singular}
% \end{frame}

\section{Cálculo y Matrices}
\begin{frame}
    \frametitle{Gradiente}
    
\end{frame}

{\setbeamercolor{palette primary}{fg=black, bg=yellow}
\begin{frame}[standout]
  Preguntas?
\end{frame}
}
{\setbeamercolor{palette primary}{fg=black, bg=yellow}
\begin{frame}[standout]
  Preguntas?
\end{frame}
}

\appendix

% \begin{frame}[fragile]{Backup slides}
%   Sometimes, it is useful to add slides at the end of your presentation to
%   refer to during audience questions.

%   The best way to do this is to include the \verb|appendixnumberbeamer|
%   package in your preamble and call \verb|\appendix| before your backup slides.

%   \themename will automatically turn off slide numbering and progress bars for
%   slides in the appendix.
% \end{frame}

% \begin{frame}[allowframebreaks]{Referencias}
% .
% \end{frame}

\end{document}
