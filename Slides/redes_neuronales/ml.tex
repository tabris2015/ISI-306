\documentclass[10pt]{beamer}

\usepackage[spanish, mexico]{babel}
\usepackage[utf8]{inputenc}

\usetheme[progressbar=frametitle]{metropolis}
\usepackage{appendixnumberbeamer}

\usepackage{booktabs}
\usepackage[scale=2]{ccicons}

\usepackage{tikz}
\def\checkmark{\tikz\fill[scale=0.4](0,.35) -- (.25,0) -- (1,.7) -- (.25,.15) -- cycle;}
\usepackage{pgfplots}
\usepgfplotslibrary{dateplot}

\usepackage{xspace}
\newcommand{\themename}{\textbf{\textsc{metropolis}}\xspace}

%%
\usepackage{color}
\definecolor{lstgrey}{rgb}{0.95,0.95,0.95}
\definecolor{mygreen}{RGB}{28,172,0} % color values Red, Green, Blue
\definecolor{mylilas}{RGB}{170,55,241}

\usepackage{listings}
\lstset{language=Python,
       backgroundcolor=\color{lstgrey},
       frame=single,
       basicstyle=\footnotesize\ttfamily,
       captionpos=b,
       tabsize=2,
  }

\lstset{language=Python,%
  %basicstyle=\color{red},
  breaklines=true,%
  morekeywords={python2tikz},
  keywordstyle=\color{blue},%
  morekeywords=[2]{1}, keywordstyle=[2]{\color{black}},
  identifierstyle=\color{black},%
  stringstyle=\color{mylilas},
  commentstyle=\color{mygreen},%
  showstringspaces=false,%without this there will be a symbol in the places where there is a space
  numbers=left,%
  numberstyle={\tiny \color{black}},% size of the numbers
  numbersep=9pt, % this defines how far the numbers are from the text
  emph=[1]{for,end,break},emphstyle=[1]\color{red}, %some words to emphasise
  %emph=[2]{word1,word2}, emphstyle=[2]{style},    
}
%

\lstset{language=C,
       backgroundcolor=\color{lstgrey},
       frame=single,
       basicstyle=\footnotesize\ttfamily,
       captionpos=b,
       tabsize=2,
  }

\lstset{language=C,%
  %basicstyle=\color{red},
  breaklines=true,%
  morekeywords={c2tikz},
  keywordstyle=\color{blue},%
  morekeywords=[2]{1}, keywordstyle=[2]{\color{black}},
  identifierstyle=\color{black},%
  stringstyle=\color{mylilas},
  commentstyle=\color{mygreen},%
  showstringspaces=false,%without this there will be a symbol in the places where there is a space
  numbers=left,%
  numberstyle={\tiny \color{black}},% size of the numbers
  numbersep=9pt, % this defines how far the numbers are from the text
  emph=[1]{for,end,break},emphstyle=[1]\color{red}, %some words to emphasise
  %emph=[2]{word1,word2}, emphstyle=[2]{style},    
}
%


\title{ISI437 - Inteligencia Artificial}
\subtitle{Aprendizaje Automático - Conceptos básicos}
\date{\today}
% \date{}
\author{Ing. Jose Eduardo Laruta Espejo}
\institute{Universidad La Salle - Bolivia}
% \titlegraphic{\hfill\includegraphics[height=1.5cm]{logo.pdf}}

\begin{document}

\maketitle

\begin{frame}[allowframebreaks]{Contenido}
  \setbeamertemplate{section in toc}[sections numbered]
  \tableofcontents[]
\end{frame}

%%%

\section{Aprendizaje Automático}
\begin{frame}
  \frametitle{Introducción}

  \begin{itemize}
    \item Existen patrones dentro de los datos que usamos a diario.
    \item Usualmente no somos capaces de visualizar información oculta o en muchas dimensiones.
    \item A veces, se necesitan modelos abstractos sobre la naturaleza de los datos.
  \end{itemize}

\end{frame}

\begin{frame}
  \frametitle{Introducción}

  A diferencia de los enfoques anteriormente estudiados, el aprendizaje automático 
  o \textit{machine learning} tiene que ver con la Inteligencia relacionada a la memoria, 
  que está asociada a datos y no a cálculos (simulación)\footnote{Nota: en el caso del aprendizaje por refuerzo, se toma un enfoque híbrido con una mezcla de simulación y datos}.

\end{frame}

\begin{frame}
  \frametitle{Tipos de aprendizaje automático}
  En general, se puede dividir el estudio del aprendizaje automático en 3 principales categorías:
  \begin{itemize}
    \item Aprendizaje Supervisado.
    \item Aprendizaje no Supervisado.
    \item Aprendizaje por refuerzo.
  \end{itemize}
  
  Se detallará cada una de estas categorías a continuación.

\end{frame}

\section{Tipos de aprendizaje automático}
\begin{frame}
  \frametitle{Ejemplo:}

  %foto de bebe

\end{frame}
\subsection{Aprendizaje supervisado}
\begin{frame}
  \frametitle{Aprendizaje supervisado}
  \begin{itemize}
    \item Conjunto de datos etiquetados previamente.
    \item Mientras más datos, mejores son los resultados.
    \item Se busca
  \end{itemize}

\end{frame}
\subsection{Aprendizaje no supervisado}
\subsection{Aprendizaje por refuerzo}
\subsection{Regresión}


{\setbeamercolor{palette primary}{fg=black, bg=yellow}
\begin{frame}[standout]
  Demo Time
  \begin{itemize}
    \item \href{http://math.hws.edu/eck/jsdemo/jsGeneticAlgorithm.html}{demo1}
    \item \href{https://rednuht.org/genetic_walkers/}{demo2}
    \item \href{https://rednuht.org/genetic_cars_2/}{demo3}
  \end{itemize}
\end{frame}
}

{\setbeamercolor{palette primary}{fg=black, bg=yellow}
\begin{frame}[standout]
  Preguntas?
\end{frame}
}

\appendix


\end{document}
